% From mitthesis package
% Version: 1.01, 2023/06/19
% Documentation: https://ctan.org/pkg/mitthesis
%
% The abstract environment creates all the required headers and footnote. 
% You only need to add the text of the abstract itself.
%
% Approximately 500 words or less; try not to use formulas or special characters
% If you don't want an initial indentation, do \noindent at the start of the abstract

% \jztodo{Each thesis must include an abstract of generally no more than 500 words single-spaced. The abstract should be thought of as a brief descriptive summary, not a lengthy introduction to the thesis. The abstract should immediately follow the title page.}

Understanding the interaction between weather and disruptions in complex air transportation network is important to the design and evaluation of preemptive measures and responses taken by air traffic managers. However, the occurrence of disruptive weather events is often rather limited compared to the amount of data available for nominal operations.  Additionally, in large-scale systems with many known and unknown confounding factors, it can be difficult to identify the relevance of existing data to different underlying distributions of interest. Furthermore, existing work generally follows a frequentist paradigm in predicting disruptions based on weather, and does not easily lend itself to inferring the causes of disruptions, which can be important both in building models and using them to make predictions, and generate test cases to stress-test proposed design decisions. In this thesis, we develop a hierarchical Bayesian model for air traffic network operations, and investigate methods for learning these models in data-constrained settings, by extend existing work on retrospectively analyzing failures. We also include a guiding case study performed on LaGuardia Airport, in which a generative model is developed for the interaction between weather conditions and airport-level parameters within a single airport, trained on unlabeled historical data, and evaluated by simulating disruptions on historical schedules. 

